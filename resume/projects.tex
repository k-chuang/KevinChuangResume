%-------------------------------------------------------------------------------
%	SECTION TITLE
%-------------------------------------------------------------------------------
\cvsection{Projects}


%-------------------------------------------------------------------------------
%	CONTENT
%-------------------------------------------------------------------------------
\begin{cventries}
%---------------------------------------------------------
%   \cventryprojects
%     {Wikipedia Article Recommender} % Organization/group
%     {Tools: Python, Gensim, NumPy, Scikit-learn} % Date(s)
%     {
%       \begin{cvitems} % Description(s) of experience/contributions/knowledge
%         % \item {Collobrated with team}
%         \item {Collaborated with team to create a Chrome plugin to recommend Wikipedia articles based on a highlighted sequence of text}
%         \item {Utilized naive cosine similarity and doc2vec algorithm to implement recommendation system}
%         \item {Key Takeaways: Team Communication, Text Mining, Paragraph Vector, Top-K Accuracy}
%         % \item {Experimented with different text preprocessing techniques}
%       \end{cvitems}
%     }

%---------------------------------------------------------
%   \cventryprojects
%     {News Article Clustering} % Organization/group
%     {Tools: Python, scipy, pandas, numpy, seaborn, scikit-learn} % Date(s)
%     {
%       \begin{cvitems} % Description(s) of experience/contributions/knowledge
%         \item {Implemented a DBSCAN clustering algorithm to cluster similar news articles}
%         \item {Ranked \textbf{3rd} in terms of NMI score}
%       \end{cvitems}
%     }

%---------------------------------------------------------
%   \cventryprojects
%     {Traffic Image Classification} % Organization/group
%     {Tools: Python, scipy, pandas, numpy, seaborn, scikit-learn} % Date(s)
%     {
%       \begin{cvitems} % Description(s) of experience/contributions/knowledge
%         \item {Designed a predictive model to predict objects from image features}
%         \item {Ranked \textbf{2nd} in terms of F1 score}
%       \end{cvitems}
%     }

%---------------------------------------------------------
  % \cventryprojects
  %   {\href{https://github.com/nguyensjsu/fa19-281-team-red-1}{Yet Another URL Shortener}} % Organization/group
  %   {Takeaways: Microservices, AKF Scale Cube, System Design} % Date(s)
  %   {
  %     \begin{cvitems} % Description(s) of experience/contributions/knowledge
  %      \item {Collaborated with team in hackathon to design, implement, and launch URL shortener web service that satisfies \textbf{AKF Scale Cube} with \textbf{React.js} frontend, \textbf{Kong} API Gateway, multiple load balancers, \textbf{Golang} API microservices, and sharded \textbf{MongoDB} clusters hosted on \textbf{AWS}}
  %       % \item {Implemented two API microservices using \textbf{Golang}, \textbf{Docker}, and \textbf{MongoDB} for most popular domains and user history}
  %     \end{cvitems}
  %   }

%---------------------------------------------------------
%   \cventryprojects
%     {NVIDIA AI City Challenge 2019} % Organization/group
%     {Tools: Python, Tensorflow, OpenCV} % Date(s)
%     {
%       \begin{cvitems} % Description(s) of experience/contributions/knowledge
%       \item {Implemented transfer learning with convolutional neural networks and various loss functions (\textbf{triplet loss}, \textbf{softmax loss}) to learn an embedding space for city-scale multi-camera vehicle re-identification}
%         % \item {Conducted research with professor in areas of vehicle and person re-identification}
%         % \item {Ranked in the \textbf{top 50\%} of all teams globally in terms of mAP@100 for track 2 of the challenge}
%         % \item {Key Takeaways: Transfer Learning, Triplet Loss, Batch Hard, Vehicle Re-identification}
%       \end{cvitems}
%     }

%---------------------------------------------------------
%   \cventryprojects
%     {Crowd Estimation} % Organization/group
%     {Tools: Python, Keras, OpenCV} % Date(s)
%     {
%       \begin{cvitems} % Description(s) of experience/contributions/knowledge
%         % \item {Implemented convolutional neural network architectures (MCNN, SANet) in Keras to count people, achieving similar performance as CVPR research papers}
%         \item {Implemented convolutional neural network architectures in \textbf{Keras} to count people in a given image \& achieved similar performance in terms of MAE and MSE as CVPR research papers}
%         % \item {Implemented convolutional neural network architectures (\textbf{MCNN}, \textbf{SANet}) in Keras to count people in a given image \& achieved similar performance in terms of MAE and MSE as CVPR research papers}
%         % \item {Achieved similar performance in terms of MAE and MSE as CVPR research papers}
%         % \item {Key Takeaways: Research Paper Implementation, Transposed Convolution, Density Map Estimation, Instance Normalization}
%       \end{cvitems}
%     }

%---------------------------------------------------------
  \cventryprojects
    {\href{https://github.com/k-chuang/data-science-competitions}{Data Science Competitions}} % Organization/group
    {Takeaways: Data Analysis, Feature Engineering, ML Model Lifecycle} % Date(s)
    {
      \begin{cvitems} % Description(s) of experience/contributions/knowledge
        % \item {Ranked \textbf{1st} in book recommender system, \textbf{2nd} in traffic image classification, \textbf{3rd} in news article text clustering, and \textbf{5th} in medical text classification in the final leaderboard in terms of F1 score, NMI score, and RMSE}
        \item {Ranked \textbf{1st} in book recommender system, \textbf{2nd} in traffic image classification, \textbf{3rd} in news article text clustering, and \textbf{5th} in medical text classification in the final leaderboard of data science competitions using \textbf{Python (pandas, scikit-learn, matplotlib, scipy)}}
    %   \item {}
        % \item {Ranked \textbf{5th} in medical text classification, \textbf{2nd} in traffic image classification, and \textbf{3rd} in news article text clustering in the final leaderboard in terms of F1 score and NMI score}
        % \item {Ranked \textbf{1st} in Kaggle competition for book recommender system and \textbf{1st} in a graph mining in terms of RMSE}
        % \item {Key Takeaways: Text Mining, Dimensionality Reduction, Cross Validation, Ensemble Methods, Internal Cluster Validation, Recommender Systems, Collaborative Filtering}
      \end{cvitems}
    }

% %---------------------------------------------------------
%   \cventryprojects
%     {Medical Text Classification} % Organization/group
%     {Tools: Python, scipy, pandas, numpy, seaborn, scikit-learn} % Date(s)
%     {
%       \begin{cvitems} % Description(s) of experience/contributions/knowledge
%         \item {Designed a k-nearest neighbor algorithm from scratch to predict medical conditions given medical text abstracts}
%         \item {Ranked \textbf{5th} in final leaderboard in terms of F1 score}
%       \end{cvitems}
%     }

%---------------------------------------------------------
%   \cventryprojects
%     {Kaggle San Francisco Crime Classification} % Organization/group
%     {Tools: Python, AWS EC2, Pandas, NumPy, Seaborn, Scikit-learn} % Date(s)
%     {
%       \begin{cvitems} % Description(s) of experience/contributions/knowledge
%         \item {Implemented an end to end Kaggle project to predict category of crime in San Francisco given temporal \& spatial features}
%         %  \item {Utilized pandas, numpy, seaborn, \& scikit-learn to analyze, explore, \& visualize the data}
%         % \item {Ranked \textbf{top 6\%} or \textbf{94th percentile} in public leaderboard with ensemble method (boosting)}
%         \item {Ranked \textbf{top 6\%} in public leaderboard in terms of multiclass log loss with boosting ensemble method}
%         % \item {Key Takeaways: Data Visualization, Exploratory Data Analysis, Feature Engineering, Cross Validation, Ensemble Methods}
%       \end{cvitems}
%     }

%---------------------------------------------------------
% 	\cventryprojects
%     {Automatic Code Generation from Images} % Organization/group
%     {Tools: Python, Keras, OpenCV} % Date(s)
%     {
%       \begin{cvitems} % Description(s) of experience/contributions/knowledge
%         \item {Implemented an end-to-end deep neural network in Keras that converts images of websites to Bootstrap (HTML/CSS) code}
%         % \item {Key Takeaways: Convolutional Neural Networks, Recurrent Neural Networks, Image Data Generator}
%         %  \item {Trained and tuned neural network hyperparameters to achieve optimal loss and \textbf{BLEU-4 score of 0.98}}
% %         \item {Utilized Google Places API to recommend therapists nearby by providing contact information and currently availability}
% %         \item {Experimented with different CNN and RNN architectures in image captioning}
%       \end{cvitems}
%     }

%---------------------------------------------------------
% 	\cventryprojects
%     {\href{https://github.com/k-chuang/utopia-alexa-skill}{Utopia Alexa Skill}} % Organization/group
%     {Takeaways: REST API, Web Scraping, Software Testing, Code Coverage} % Date(s)
%     {
%       \begin{cvitems} % Description(s) of experience/contributions/knowledge
%         \item {Developed and deployed a serverless Alexa skill to help people with depression using \textbf{Python} and \textbf{AWS (Alexa Skills Kit, Lambda)}}
%         % \item {Key Takeaways: REST API, Web Scraping, Code Coverage, Continuous Integration, Serverless Computing}
%         %  \item {Extracted data to incorporate with features of Alexa skill from cloud and web services using web scrapers and RESTful APIs}
% %         \item {Utilized Google Places API to recommend therapists nearby by providing contact information and current availability}
% %         \item {Implemented comprehensive tests (unittest, pytest) in a CI environment (Travis CI, Codecov.io)}
%       \end{cvitems}
%     }

%---------------------------------------------------------
%   \cventryprojects
%     {Chicago Crime Data Exploration} % Organization/group
%     {Tools: Python, pandas, numpy, matplotlib, folium} % Date(s)
%     {
%       \begin{cvitems} % Description(s) of experience/contributions/knowledge
%         \item {Utilized pandas, numpy, and matplotlib to analyze, explore, \& visualize the Chicago crime data and answer critical questions}
% %         \item {Created data visualizations using matplotlib (pie/bar charts, time plots) and folium (choropleth maps, heat maps, cluster maps)}
%       \end{cvitems}
%     }

%---------------------------------------------------------
%   \cventryprojects
%     {Freesound CLI Web Scraper} % Organization/group
%     {Tools: Python, Selenium, Travis CI, Coveralls, pytest} % Date(s)
%     {
%       \begin{cvitems} % Description(s) of experience/contributions/knowledge
%         \item {Developed a web scraper to automate downloading audio files \& associated metadata from website freesound.org}
%       \end{cvitems}
%     }


%---------------------------------------------------------
\end{cventries}
